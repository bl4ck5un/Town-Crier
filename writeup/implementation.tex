\section{\tc Implementation Details}


\subsection{TC Contract}

To implement the TC Contract we implemented the version of \tcont including fees and cancellation as described in Section~\ethan{ref}.
The contract is implemented in Solidity, a high-level language with JavaScript-like syntax which compiles to Ethereum Virtual Machine bytecode---the language that can be executed as a contract on the blockchain.

In order to handle the most general type of requests---including encrypted parameters---the \tcont implementation requires two parameter fields.
The first specifies what type of request is being made (e.g. stock price or flight status).
The second is a byte array of user-specified size.
This byte array will be parsed and interpreted inside the enclave when it fulfills the request, but is treated as an opaque byte array by \tcont.

As we discuss in Section~\ethan{ref authenticity}, TC must pass the entire byte array back to {\bf Deliver} to ensure that the request has not been modified.
Unfortunately, this verification (as well as Ethereum's cost on each byte in a transaction) means the size of this byte array greatly affects the cost of calling the {\bf Deliver} entry point.
In fact, when the byte array contains 400 bytes of data, this extra cost outweighs all other costs of {\bf Deliver} combined (not including \dgcallback).
In order to achieve our desired guarantees of cost while not demanding excessive fees for requests which supply only a few bytes of data,
our implementation scales the minimum fee with the length of the user provided byte array.


\ethan{This figure really doesn't belong here. It probably belongs with the gas explanation that Elaine wrote, but we can move it later.}

\begin{figure}[h!]
\centering
\begin{tikzpicture}
  [local-entity/.style={entity,minimum height=3.5em,text width=8em}]
  \node[local-entity,trusted] (ctc) {};
  \node[local-entity,draw=none,anchor=north] (ctc-inner) at (ctc.north) {TC Contract\\$\tcont$};
  \node[local-entity,trusted,right=6em of ctc,text width=5em] (enc) {Enclave};
  \node[local-entity,fill=white,below=3.5em of ctc] (cu) {User Contract\\$\reqcont$};
  \node[local-entity,fill=red!30,below=3em of cu] (user) {User};

  \path[-stealth,color=red,ultra thick] (user) edge [transform canvas={xshift=-3.5em}] ([yshift=0.75em]cu.south);
  \path[color=red,thick] (cu.south) edge [right,transform canvas={xshift=-3.5em}] node [text=black,text width=4em,align=center,yshift=1.75em] {\footnotesize Insufficient\\[-0.25em]gas aborts\\[-0.2em]with no\\[-0.5em]effect} (ctc.south);
  \node[draw=red,cross out,minimum size=1ex,thick] () at ([xshift=-3.5em]ctc.south) {};

  \draw[-stealth,color=green!50!black,line width=0.8ex] ([yshift=-0.25em]enc.west) -- ([xshift=-1.2em,yshift=-0.25em]ctc.east);
  \draw[-stealth,color=green!50!black,line width=0.5ex] ([yshift=-0.25em]enc.west) -| ([xshift=3em,yshift=-0.9em]cu.north);
  \draw[color=green!50!black,thick,dashed]  ([xshift=3.75em]cu.north) |- ([yshift=-1.25em]ctc.east);
  \path[-stealth,color=green!50!black,ultra thick,dashed]  ([yshift=-1.25em]ctc.east) edge [below] node [text=black,text width=4em,align=center,xshift=-0.4em] {\footnotesize Extra gas\\[-0.4em]refunded} ([xshift=0.8em,yshift=-1.25em]enc.west);

  \begin{pgfonlayer}{background}
    \node[bg-box,
          blockchain-color,
          fit={($(ctc.north west)+(-0.3em,0.3em)$)($(cu.south east)+(0.3em,-0.3em)$)},
          label=above:{\bf Blockchain}] () {};
    \node[bg-box,
          tc-server-color,
          fit={($(enc.north east)+(0.3em,0.3em)$)($(enc.south west)+(-0.3em,-0.3em)$)},
          label=above:{\bf TC Server}] () {};
  \end{pgfonlayer}
\end{tikzpicture}
\caption{{\bf Gas Payment in Ethereum}
  When an Ethereum function is invoked, all operations must be paid for in gas.
  All gas is provided by the user initiating the transaction, even if that transaction includes nested function calls.
  By default, a contract calling a function passes that function all gas available to the parent call, but it has the option of limiting that gas.
  If more gas is provided than is used, all remaining gas is automatically refunded.
  If more gas is required than provided, all gas provided is expended and the function call reverts all changes and aborts.}
\label{fig:gas}
\end{figure}




\subsection{TC Server}
\subsubsection{The \medname}
\subsubsection{The \encname}

